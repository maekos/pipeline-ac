\section{Latch Etapa de Ejecuci\'on - Etapa de memoria}

Esta latch separa las etapas de ejecuci\'on y memoria.
\begin{figure}[H]
\centering
\includegraphics[scale=0.35]{img/latch_ex_m}
\caption{Latche Etapa de ejecucion - Etapa Memoria}
\label{fig:latch_ex_mem}
\end{figure}
Tiene como entradas:
\begin{itemize}
  \item \textbf{alu\_result}: Bus de 32 bits que tiene el resultado de la operaci\'on que se realizo en la ALU.
  \item \textbf{data2}: Bus de 32 bits, es pasado por esta etapa directamente desde el latch anterior para las instrucciones que tiene que escribir en la etapa de memoria justamente la memoria de datos, la escriben con este valor siempre y cuando la señal de escritura de memoria este activada.
  \item \textbf{dst}: Registro en el cual se va a escribir en el banco de registros.
  \item \textbf{clk}: Reloj general del sistema.
  \item \textbf{mem\_to\_reg}: Señal que activa la toma de datos de la memoria hacia los registros.
  \item \textbf{mem\_write}: Señal que habilita la escritura en memoria.
  \item \textbf{reg\_write}: Señal que habilita la escritura de los registros.
  \item \textbf{rst}: Señal que pone a cero todos los registros del m\'odulo.
\end{itemize}

Las salidas son las mismas que las entradas anteriores, salvo la señal de reset y la de clock.